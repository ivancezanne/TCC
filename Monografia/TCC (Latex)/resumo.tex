	O processo de formação de imagens consiste em projetar regiões visíveis de uma cena em um plano bidimensional, desta forma consegue-se registrar um ambiente tridimensional em um outro bidimensional. Na bibliografia sobre computação gráfica encontram-se vários métodos desta captura, de forma digital. Neste trabalho propõe-se uma técnica capaz de criar uma nuvem de pontos tridimensionais que represente um objeto ora apresentado por uma única imagem bidimensional. Utilizando imagens de apelo histórico à computação gráfica e à arte, apresenta-se uma forma de identificação de um sistema de coordenadas de referência particular a cada cena representada na imagens, de forma que seja possível o estudo analítico da constituição espacial dos objetos de interesse na cena. Referenciadas as cenas, pôde-se obter uma transformação projetiva que implemente uma correspondência entre pontos 2D de uma imagem com pontos 3D de um sistema de referências. Por fim, obteve-se um software que permite um referenciamento aproximado da cena, implicando em uma reconstrução coesa, porém, com certas limitações.

\noindent Palavras-chaves: Visão Computacional, Processamento Digital de Imagens, Reconstrução 3D