	O processo de gravação de imagens fotográficas consiste em projetar regiões visíveis de uma cena em um plano bidimensional, desta forma consegue-se registrar um ambiente tridimensional em um outro bidimensional. Na bibliografia sobre computação gráfica encontram-se vários métodos desta captura, de forma digital. Neste trabalho propõe-se uma técnica capaz de criar uma malha poligonal tridimensional que represente completamente um objeto ora apresentado por uma imagem bidimensional. Utilizando imagens de apelo histórico à computação gráfica e à arte, apresenta-se uma forma de identificação de um sistema de coordenadas de referência já existente e particular a cada objeto representado na imagens, de forma que seja possível o estudo analítico de sua constituição espacial. Uma vez bem referenciados, pode-se obter uma transformação linear ou vetorial que implemente uma correspondência entre pontos 2D de uma imagem com pontos 3D de um sistema de referências. Dada a particularidade das imagens estudadas, é possível inferir padrões de simetria e proporcionalidade nos objetos retratados, permitindo a representação, também, de regiões do objeto oclusas nas imagens.

\noindent Palavras-chaves: Processamento Digital de Imagens, Computação Gráfica, Reconstrução 3D