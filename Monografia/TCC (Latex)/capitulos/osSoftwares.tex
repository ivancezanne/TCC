\chapter{Softwares Desenvolvidos}
	\label{capituloSoftwares}

	Ao decorrer deste trabalho, fora o software que implementa a técnica de reconstrução descrita nesta monografia, foi sendo necessário que técnicas adjuntas à estudada aqui também fossem implementadas em softwares a fim de facilitar comprovações de resultados e automatizações de tarefas auxiliares ao projeto; além de agregar determinada riqueza de esforço e de ferramentas a este trabalho de conclusão de curso.
	
	A seguir são apresentados os exemplos mais representativos dos softwares desenvolvidos ao longo do projeto.
	
	\section{Aplicativo de detecção de retas}
		\label{appDeteccaoRetas}
		Tanto para auxiliar na detecção automática dos segmentos de reta do sistema de coordenadas de referência quanto dos pontos (interseção de segmentos de reta) de interesse à reconstrução. Foi desenvolvido um aplicativo para a obtenção de parâmetros analíticos de retas apresentadas em uma região de uma imagem digital.
		
		As linhas inferidas como interessantes, por serem representadas através da rasterização - processo de conversão da representação vetorial para a matricial \cite{compGrafTeoPrat} - precisaram sofrer um processo de regressão linear a fim de se obter uma aproximação dos parâmetros geométricos analíticos de sua formação. A regressão linear executada para a obtenção dos coeficientes angular e linear das retas de interesse se deu através do método dos mínimos quadrados.
	
	O Métodos dos Mínimos Quadrados é uma técnica de regressão linear que busca ajustar corretamente coeficientes de uma curva algébrica de forma que estes minimizem o somatório do quadrado da diferença entre os valores dos dados reais e os valores estimados, minimizando o erro de aproximação.
	
	Com uma interface apropriada, pôde-se carregar uma imagem e, através da demarcação manual de uma região de interesse na mesma (demarcada por um retângulo vermelho), pontos da possível linha foram detectados pela sua cor preta e submetidos ao método dos mínimos quadrados (figura \ref{figMinQuadInterface}).
	
	Após a aproximação são exibidos, na parte superior da janela, os coeficientes encontrados e é plotada, sobre a imagem, a reta encontrada.
	
	\begin{figure}[!htb]
		\centering
		\includegraphics[height=6cm]{imagens/printAppMinimosQuadrados.png}
		\caption{O programa de regressão de rasterização com uma aproximação de exemplo}
		\label{figMinQuadInterface}
	\end{figure}

	Posteriormente, os coeficientes encontrados foram submetidos ao programa Microsoft Mathematics\textregistered  comprovando que os valores encontrados são coerentes (figura \ref{mathematics}). Note que, como o sentido de orientação dos pontos da imagem (de cima para baixo) é inverso ao do plano cartesiano (de baixo para cima) o aspecto da reta em relação ao aspecto da linha é invertido.
	
	\begin{figure}[!htb]
		\centering
		\includegraphics[height=5cm]{imagens/mathematics1.png}
		\caption{Reta analítica correspondente à linha da região demarcada na figura \ref{figMinQuadInterface}}
		\label{mathematics}
	\end{figure}
	
	Com esta técnica se obteve uma possibilidade de automatização da busca de elementos de interesse nas imagens estudadas.
	
	Este aplicativo está disponível em \url{https://github.com/ivancezanne/TCC/tree/master/App_MinimosQuadrados}.
	
	\section{Aplicativo de Projeção 3D}
		\label{app3D}
		
		Tendo em vista que o resultado deste trabalho é gerar malhas 3D, um aplicativo (figura \ref{imagemApp3D}), baseado na descrição presente em \cite{compGraphsPrincPrat3ed} e capaz de interpretar aquivos PLY \cite{ply}, foi desenvolvido. Mesmo apresentando uma limitação quanto a definições de usuário em relação a cores e normais de faces, a utilidade deste aplicativo no contexto deste trabalho foi notável.
		
		\begin{figure}[!htb]
			\centering
			\includegraphics[height=5cm]{imagens/printApp3D.png}
			\caption{Aplicativo de projeção 3D com a famosa malha Suzanne}
			\label{imagemApp3D}
		\end{figure}
		
		Além da projeção da malha 3D descrita em um arquivo PLY de entrada, este aplicativo possibilita manipulações geométricas na malha (translação e rotação, apenas) e algoritmos prontos de normalização (alteração proporcional das dimensões para o máximo de 1 unidade de medida) e centralização (translação do baricentro da malha à origem do espaço 3D).
		
		Como auxílios visuais o aplicativo também permite escolher quais elementos da malha deverão ser projetados à visualização (figura \ref{imagemApp3DRenderizar}) e alterar as cores dos elementos que compõem a malha (figura \ref{imagemApp3DCores}).
		
		\begin{figure}[!htb]
			\centering
			\subfloat[Menu onde se escolhe quais elementos da malha serão renderizados]{
				\includegraphics[height=5cm]{imagens/printApp3D-renderizar.png}
				\label{imagemApp3DRenderizar}
			}
			\quad
			\subfloat[Menu onde se pode alterar as cores dos elementos da malha]{
				\includegraphics[height=5cm]{imagens/printApp3D-cores.png}
				\label{imagemApp3DCores}
			}
			\caption{Opções adicionais do aplicativo de projeção 3D de auxílio à visualização}
			\label{imagemApp3DAuxiliosVisuais}
		\end{figure}
		
		Em testes com nuvens de pontos pós-tratadas (após a inserção de faces), notou-se uma limitação do componente gráfico da linguagem C\# usado na renderização da projeção - o \textit{picturebox}. Tal limitação impede que faces parcialmente fora da área de visualização sejam renderizadas corretamente, causando uma distorção nos polígonos planares correspondentes às faces parcialmente visíveis. Todavia, uma vez que vértices não apresentavam tal problema, as funcionalidades implementadas neste aplicativo o fizeram preferível, na visualização de resultados brutos deste trabalho, em relação ao \textit{software} MeshLab.
		
		Este aplicativo pode ser encontrado em \url{https://github.com/ivancezanne/TCC/tree/master/App_Projecao}.
		
	\section{Aplicativo de Reconstrução}
		\label{appReconstrucao}
		
		A técnica de reconstrução deste trabalho (capítulo \ref{capituloReconstrucao}) foi implementada através de um \textit{software} (figura \ref{printAppReconstrucao}) que, com forte auxílio visual, busca otimizar a reconstrução almejada.
		
		\begin{figure}[!htb]
			\centering
			\includegraphics[height=4cm]{imagens/printAppReconstrucao.png}
			\caption{Tela inicial do aplicativo de reconstrução}
			\label{printAppReconstrucao}
		\end{figure}
		
		Inicialmente, o software necessita que uma imagem seja aberta em sua interface para que as demais funcionalidades sejam disponibilizadas. Com uma imagem carregada (que já é aberta com um conjunto de referências arbitrário), a demarcação de pontos de referência e de interesse (capítulo \ref{capituloReconstrucao}) pode ser feita através de \textit{clicks} na região desejada da imagem (figura \ref{printAppReconstrucaoDemarcacao}).
		
		\begin{figure}[!htb]
			\centering
			\subfloat[Interface apenas com uma imagem incial]{
				\includegraphics[height=4cm]{imagens/printAppReconstrucaoInicio.png}
				\label{AppReconstrucaoInicio}
			}
			\enskip
			\subfloat[Após o \textit{click}, um \textit{prompt} solicita dados acerca do ponto]{
				\includegraphics[height=4cm]{imagens/printAppReconstrucaoPrompt.png}
				\label{AppReconstrucaoPrompt}
			}
			\enskip
			\subfloat[Pelos \textit{radio buttons}, o mesmo \textit{prompt} é utilizado para catalogar pontos de referência e de interesse]{
				\includegraphics[height=4cm]{imagens/printAppReconstrucaoPromptInteresse.png}
				\label{AppReconstrucaoPromptInteresse}
			}
			\enskip
			\subfloat[Uma cena completamente catalogada]{
				\includegraphics[height=4cm]{imagens/printAppReconstrucaoPronto.png}
				\label{AppReconstrucaoPronto}
			}
			\caption{Utilização da interface para a demarcação de pontos de referência e interesse}
			\label{printAppReconstrucaoDemarcacao}
		\end{figure}
		
		Após o completo fornecimento de dados para a reconstrução, é necessário que se execute a calibração de câmera e só então a reconstrução será possível de ser executada. Para a implementação da calibração foi utilizada o método de cálculo de autovalores e autovetores, baseado no algoritmo QR, da biblioteca ALGLIB Free Edition v3.8.2 para linguagem C\#.
		
		Ao início da reconstrução, os dados descritos no capítulo \ref{capituloReconstrucao} (tolerância projetiva, incremento e critério de convergência) são requisitados pelo software e seu fornecimento é obrigatório ao usuário (figura \ref{printAppReconstrucaoReconstrucao}).
		
		\begin{figure}[!htb]
			\centering
			\includegraphics[height=5cm]{imagens/printAppReconstrucaoReconstrucao.png}
			\caption{\textit{Prompt} com dados necessários à reconstrução}
			\label{printAppReconstrucaoReconstrucao}
		\end{figure}
		
		Em adição aos critérios arbitrários de convergência descritos no capítulo \ref{capituloReconstrucao}, o software oferece a possibilidade de escolha manual, pelo usuário, do valor da coordenada Z dentre os que o algoritmo já encontrou como possíveis (figura \ref{printAppReconstrucaoEscolhaManual})
		
		\begin{figure}[!htb]
			\centering
			\includegraphics[height=5cm]{imagens/printAppReconstrucaoEscolhaManual.png}
			\caption{\textit{Prompt} para escolha manual da coordenada Z}
			\label{printAppReconstrucaoEscolhaManual}
		\end{figure}
		
		Em posse disto a técnica de reconstrução proposta neste trabalho pode ser executada por qualquer usuário interessado através de um interface adequada. Todavia, ferramentas adicionais como visualização de aproximação após calibração de câmera; salvamento e carregamento de pontos de referência e de interesse através de arquivos; e captura de tela de trabalho foram implementadas tornando o uso do software mais completo e eficiente.
		
		Este aplicativo de reconstrução está disponível em \url{https://github.com/ivancezanne/TCC/tree/master/App_Reconstrucao}.
		