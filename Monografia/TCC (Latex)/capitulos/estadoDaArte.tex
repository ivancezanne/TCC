\chapter{Estado da Arte}
	\label{capituloEstadoDaArte}

	Com o uso de \textit{hardwares} elaborados (câmeras sincronizadas ou câmeras de percepção volumétrica), definição estrita de objeto de reconstrução e distintas vertentes geométricas, a reconstrução 3D já é uma prática difundida; sendo possível encontrar um grande número de artigos acadêmicos sobre o assunto, além de trabalhos de empresas de entretenimento famosas que implementam reconstrução 3D, cada um de uma forma particular.

	Neste capítulo é exposta uma breve coleção de exemplos representativos desta difusão da reconstrução a fim de configurar o cenário onde este trabalho de conclusão de curso se encaixa.
	
	\section{Múltiplas Visões e Geometria Epipolar}
	
	Uma abordagem muito comum para a implementação de um sistema de reconstrução 3D é o uso de diferentes pontos de vista sobre um mesmo objeto. A aquisição de dados de um objeto sob diferentes pontos de vista pode ser feita através de sistemas com múltiplos dispositivos de captura (câmeras) devidamente sincronizados e calibrados ou com imagens já obtidas, desde que seja conhecido um certo número de dados tridimensionais e dados de correspondência. Câmeras utilizadas para obtenção de múltiplas visões de um objeto são conhecidas como câmeras estéreo e tal técnica baseada em múltiplas visões, de estereoscopia.
	
	Observando relações de correspondência entre pontos de distintas imagens, o uso da geometria epipolar fornece meios de concluir uma localização tridimensional a partir de dados bidimensionais. De forma sucinta, a geometria epipolar utiliza a técnica de triangulação entre os centros de projeção - que são pontos tridimensionais - de duas câmeras e o ponto de interseção - que também é tridimensional e pode ser chamado de epipolo - entre as retas formadas pelos centros de projeção das câmeras e o ponto da imagem em análise \cite{animation}.
	
	\begin{figure}[!htb]
		\centering
		\includegraphics[height=5cm]{imagens/geometriaEpipolar.png}
		\caption{Triangulação utilizada na geometria epipolar}
		\label{fotoGeometriaEpipolar}
	\end{figure}
	
		\subsection{Cinema 3D através de Estereoscopia}
		
		Em \cite{disneyRelatedWork} vê-se a sugestão de um complexo sistema motorizado de câmeras de vídeo com um ajuste automático de parâmetros de calibração para maior precisão na determinação de características tridimensionais obtidas através do sistema de câmeras estéreo proposto.
		
		Esta abordagem é efetiva para os autores dada a sua finalidade cinematográfica; e já que filmes em 3D são cada vez mais comuns nas salas de cinema em todo mundo, o investimento feito para obter a precisão que este trabalho alcança é facilmente compensado.
		
		Em um contexto mais amplo, que não suporta uma infra-estrutura complexa de uma filmagem dos estúdios de cinema internacionais ou demanda um dinamismo maior, o uso de um sistema motorizado de câmeras em sincronia é uma estratégia pouco eficiente, apesar da sua grande eficácia.
		
		\subsection{Um Abordagem Mais Simples da Estereoscopia}
		
		\cite{stereoRelatedWork} utiliza uma abordagem mais enxuta da reconstruçao 3D através de estereoscopia, sob a premissa de que as dimensões do objeto a ser reconstruído são conhecidas. Tal premissa, mesmo válida para o trabalho citado, reduz drasticamente o escopo da técnica sugerida.
		
		Usando um único par de imagens estereoscópicas estáticas e as dimensões do objeto estudado, ainda é preciso efetuar alguns processos de normalização a ajuste algébrico para que condições de convergência, como o epipolo, possam ser obtidas. Tais empecilhos, inclusive, são decorrentes de uma análise do objeto em termos de coordenadas absolutas, o que é irrelevante para a análise de um objeto isolado cuja reconstrução não faz parte de um cenário igualmente importante.
		
		Assim, a reconstrução 3D de \cite{stereoRelatedWork} ainda não se adapta completamente ao objetivo deste trabalho.
		
	\section{Calibração de Câmera e Geometria Projetiva}
		\label{secaoEstadoDaArteCalibracao}
	
		Um conceito conhecido na computação gráfica é o de câmera virtual - um modelo matemático computacional que representa o funcionamento de um dispositivo de captura \cite{fundCompGraf}. 
		
		A técnica de calibração de câmera consiste em estimar os parâmetros de uma câmera virtual capaz de produzir uma imagem. Uma vez que imagens são produzidas através de projeções, com o uso da geometria projetiva é possível haver dedução, sob certas hipóteses, de dados 3D após a análise de uma única imagem 2D.
		
		Em \cite{3DFromLineDrawings}, após um processo de obtenção dos parâmetros de uma câmera, é possível transformar polígonos pertencentes ao plano da imagem em polígonos equivalentes pertencentes a planos tridimensionais distintos do plano da imagem. Infortunadamente, esta referência não fora conhecida tão cedo quanto desejado, impedindo que este trabalho pudesse ter maior embasamento e robustez baseado em tal referência. No entanto, trabalhos parecidos sobre calibração de câmera para reconstrução 3D, que, estes sim, serviram como embasamento inicial, estão presentes em \cite{juizVirtual} e \cite{szenbergDoutorado}