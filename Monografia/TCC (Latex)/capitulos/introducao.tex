\chapter{Introdução}
	\label{introducao}
	\pagenumbering{arabic}
	\setcounter{page}{1}

	\section{Justificativa}

	O processo fotográfico, surgido no século XIX, determinou um marco na forma de registrar a realidade humana, sendo capaz de capturar a luz proveniente do ambiente e convertê-la em regiões semelhantes sobre uma superfície manuseável chamada de fotografia. Tal processo tem sido motivo de estudo de diversas áreas do conhecimento e, hoje, já é possível executar tal processo, outrora baseado em películas fotossensíveis, de forma digital e computacional.
	
	Uma das áreas de estudo das imagens digitais é a visão computacional - campo da ciência da computação que discute acerca da interpretação e extração de conhecimento de imagens. Desde a década de 1960, a visão computacional apresenta trabalhos interessantes acerca de algoritmos de emulação de visão humana e interpretação de imagens, vide \cite{firstCVWork}. Derivado disto, hoje é possível a construção de ferramentas úteis para a compreensão via \textit{software} de cenas retratadas em imagens.
	
	Fato é que, o processo de formação de uma imagem digital funciona baseado em um processo de projeção de regiões visíveis de um ambiente tridimensional - o universo real - em uma superfície plana bidimensional - a imagem. Facilmente se percebe que tal técnica consiste em reduzir em uma dimensão o meio de representação de objetos de uma cena. Apesar de avanços tecnológicos neste processo permitirem a reprodução fiel das propriedades da luz, facilitando a noção do observador da imagem acerca  dos aspectos físicos tridimensionais (dimensões, distâncias etc.) da cena retratada, os dados que representam tais aspectos foram truncados. Tais dados tridimensionais truncados podem apresentar utilidade para áreas da sociedade que têm nas imagens armazenadas sua fonte de conteúdo. Áreas como conservação e restauro de patrimônio arquitetônico, serviços de GPS e mapas e investigação de cenários reais trabalham constantemente com representações bidimensionais de objetos tridimensionais (edifícios, avenidas urbanas, cômodos de residências mobiliados etc.) onde o sucesso de suas tarefas depende da compreensão espacial do seu objeto de estudo.
	
	De tal forma, a existência de um processo digital reverso ao processo da projeção, capaz de reconstruir objetos 3D a partir de imagens 2D, representa um avanço significativo na capacidade de se representar a realidade em forma digital.
	
\section{Problemática}

	O \textit{VW Beetle} de Sutherland entrou para a história da computação gráfica por ser o primeiro objeto real digitalizado em três dimensões. Em um processo liderado pelo professor Ivan Edward Sutherland, um grupo de alunos demarcou manualmente, com giz, sobre a carroceria de um veículo Volkswagen Beetle 1967, pontos de interesse para a construção do modelo 3D do carro \cite{MappingSutherlandVW}.
	
	Tendo a experiência sido bem sucedida, um modelo tridimensional foi obtido. No entanto a manutenção dos dados deste modelo é nebulosa, não se sabe ao certo onde obter uma versão correta deste modelo ao passo que já se questiona se os seus dados originais estão preservados para fins de acesso público.
	
	No entanto, a própria Universidade de Utah - sede do experimento - mantém as imagens deste modelo, sob a curadoria do \textit{Computer History Museum}, além de existirem registros individuais com parâmetros úteis a este trabalho (vide capítulo \ref{materiaisEMetodos}). Assim, há a possibilidade de se restaurar os dados tridimensionais deste modelo 3D digital icônico.
	
	Em um caso similar, o Bule de Leite de Newell é, originalmente, parte integrante de um famoso modelo tridimensional chamado \textit{Teaset} de autoria do pesquisador Martin Edward Newell. O modelo \textit{Teaset} pode ser encontrado facilmente em repositórios na \textit{internet}, no entanto, em nenhuma das fontes consta o bule de leite. Sabe-se que tal objeto era parte integrante do conjunto \textit{Teaset} por abordagens e imagens de trabalhos do próprio autor, \cite{newellResearch}. 
	
	Por tal circunstância, percebe-se que o Bule de Leite de Newell também se trata de um elemento de apelo histórico ao campo de estudo deste trabalho cujo conhecimento é interessante e não tem seus dados tridimensionais publicados.
	
	Em última instância, há uma litografia datada do século XVII de autoria do artista Albrecht Dürer que representa uma cena onde os personagens exemplificam a implementação da projeção perspectiva (vide seção \ref{secaoProjecoes} sobre as projeções). Tal cena, por ser uma documentação pioneira do estudo da perspectiva, apresenta uma oportunidade de trabalhar sob dados antigos e fornecer uma previsão de como ampliar este trabalho de conclusão de curso para a reconstrução 3D de cenários com mais de um objeto.
	
\section{Objetivos}

	Em face da problemática disposta, este trabalho visa, em âmbito específico, obter um conjunto de pontos tridimensionais que represente o \textit{VW Beetle} de Sutherland, o Bule de Leite de Newell e alguns elementos da cena de Dürer, de forma que seja possível a visualização destes objetos em um software de visualização 3D. \\
	Para a obtenção de sucesso no objetivo geral, os seguinte objetivos específicos foram estabelecidos:
	\begin{itemize}
		\item Definir um método de identificação de sistemas de coordenadas de referências nas imagens estudadas;
		\item estabelecer uma correspondência entre pontos das imagens com pontos tridimensionais; e
		\item inferir informações de padronização deste objetos, tais como simetrias e repetições.
	\end{itemize}